% add new glossaryentries here...
% use in tex with \gls{label}

\newglossaryentry{sqlite}{
	name=SQLite,
	description={SQLite ist eine Datenbankengine, welche ohne Konfiguration auskommt. Es handelt sich dabei um eine Datenbank in einer Datei},
	first={SQLite}
}

\newglossaryentry{php}{
	name=PHP,
	description={ PHP (Hypertext Preprocessor) ist eine Webprogrammiersprache welche auf dem Server ausgeführt wird und in der Regel eine dynamische HTML Webseite generiert.
	},
	first={PHP}
}

\newglossaryentry{mariadb}{
	name=MariaDB,
	description={ MariaDB ist eine Weiterentwicklung der berühmten MySQL Datenbank, sie ist Quelloffen und verfügt über hohe Kompatibilität mit MySQL, für weitere Informaitonen sei auf den ausführlichen Artikel auf Wikipedia verwiesen. \url{http://en.wikipedia.org/wiki/Mariadb}, aufgerufen am 07.05.2013
	},
	first={MariaDB}
}

\newglossaryentry{webframework}{
	name=Webframework,
	description={ Ein Webframework ist eine Entwicklungsbasis für die Softwareentwicklung von Webapplikationen, also serverbasierten Applikationen, welche über Webbbrowser aufgerufen werden. Das Framework übernimmt dabei grundlegende Funktionen und beschleunigt dabei die Etnwicklung massiv.},
	first={Webframework}
}

\newglossaryentry{api}{
	name=API,
	description={ Die API (Application Programming Interface), beschreibt die Schnittstelle zu einem System oder einer Technologie. },
	first={API}
}

\newglossaryentry{git}{
	name=git,
	description={ Das Versionscontrollsystem git ist ein dezentrales OpenSource Sourcecodeversionierungssystem, welches sich für die Entwicklung von Software in Teams sehr gut eignet.},
	first={git}
}