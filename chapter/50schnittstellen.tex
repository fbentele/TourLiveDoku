\chapter{Schnittstellen}

Das Aufnahmesystem kommuniziert mit den Servern in beide Richtungen, wobei jeweils das Aufnahmesystem aktiv Daten liefert und sich neue Daten holt. In diesem Kapitel werden die Schnittstellen zwischen Aufnahmesystem und Tourlive Server sowie Devicemanagement Portal.

\section{TourLive Server Schnittstelle}

TODO FLO

\section{Devicemanagement Schnittstelle}

Für alle Funktionen, auf die von dem Aufnahmesystem her zugegrifen werden, steht eine JSON Schnittstelle mit URL zur Verfügung. Im folgenden werden diese näher beschrieben.

\subsection{Status posten und Einstellungen erhalten}

Um den aktuellen Status des Aufnahmesystems dem Server mitzuteilen kann diese Methode aufgerufen werden. Als Antwort erhält man die im Moment aktuellen Einstellungen.

{\bf URL: }http://tlng.cnlab.ch/devmgmtsrv/api/getdevicemanagementcontainer 

\subsubsection{Status JSON}

Das JSON eines Status Posts sieht folgendermassen aus:

\begin{figure}[H]
	\centering
	\caption{Status JSON}
	\lstinputlisting[language=json]{jsonfiles/status.json}
\end{figure}


\subsubsection{Einstellungen JSON}

Die Antwort die man bei dieser Anfrage erhält ist die folgende:

\begin{figure}[H]
	\centering
	\caption{Einstellungen JSON}
	\lstinputlisting[language=json]{jsonfiles/settings.json}
\end{figure}


\subsection{Log posten}

Um neue Logeinträge dem Server zu übermitteln kann diese Methode benutzt werden.

{\bf URL: }http://tlng.cnlab.ch/devmgmtsrv/api/postlog

\subsubsection{Log JSON}

Das JSON eines Log Posts sieht folgendermassen aus:

\begin{figure}[H]
	\centering
	\caption{Log JSON}
	\lstinputlisting[language=json]{jsonfiles/log.json}
\end{figure}

\subsection{Nachricht abholen}

Sofern eine neue Nachricht für das Gerät vorhanden ist kann sie über diese Methoden abgeholt werden.

{\bf URL: }http://tlng.cnlab.ch/devmgmtsrv/api/getmsg/{deviceId}

\subsubsection{Nachrichten JSON}
Das JSON für die Nachrichten ist das folgende:
\begin{figure}[H]
	\centering
	\caption{Nachrichten JSON}
	\lstinputlisting[language=json]{jsonfiles/message.json}
\end{figure}


