\chapter{Schnittstellen}

Das Aufnahmesystem kommuniziert sowohl mit dem TourLive Server als auch mit dem Device Management Server in beide Richtungen. In diesem Kapitel werden die Schnittstellen zwischen Aufnahmesystem, TourLive Server sowie Device Management Server und die öffentliche Schnittstelle erläutert.

\section{TourLive Interne Schnittstelle}
\label{sec:tourliveserverapi}
Die Aufnahmegeräte liefern an dieser Stelle die ValueContainer, Bilder und Videos dem TourLive Server, dies wird als die interne Schnittstelle (\textit{\gls{api}}) bezeichnet. Die benötigten Felder sind im Bereich \textit{Request} erkennbar.
{\renewcommand{\arraystretch}{1}
    \begin{longtable}{ p{2.5cm} p{3.5cm} p{6cm}}
	\textbf{URL:} & \multicolumn{2}{p{10cm}}{/api/valuecontainer} \\
	\textbf{Request:} & Method: & POST \\
		& Content-Type: & application/json \\
		& Body: & JSON-String 'ValueContainer'\\
	\textbf{Response:} & Header: & HTTP/1.1 200 OK \\
		& Body: & <empty> \\
	\textbf{Zeitpunkt:} & \multicolumn{2}{p{10cm}}{Bei laufender Aufnahme} \\ 
	\textbf{Interval:} & \multicolumn{2}{p{10cm}}{einstellbar} \\
\hline
\hline
	\textbf{URL:} & \multicolumn{2}{p{10cm}}{/api/image} \\
	\textbf{Request:} & Method: & POST \\
		& Content-Type: & multipart/form-data [3 Boundaries] \\
		& Body-Boundary 1: & text/plain (timestamp) \\
		& Body-Boundary 2: & text/plain (deviceId) \\
		& Body-Boundary 3: & application/octet-stream (Bild) \\
	\textbf{Response:} & Header: & HTTP/1.1 200 OK \\
		& Body: & <empty> \\
	\textbf{Zeitpunkt:} & \multicolumn{2}{p{10cm}}{Bei laufender Aufnahme, falls Image aktiviert ist} \\ 
	\textbf{Interval:} & \multicolumn{2}{p{10cm}}{einstellbar} \\
\hline
\hline
	\textbf{URL:} & \multicolumn{2}{p{10cm}}{/api/video} \\
	\textbf{Request:} & Method: & POST \\
		& Content-Type: & multipart/form-data [3 Boundaries] \\
		& Body-Boundary 1: & text/plain (timestamp) \\
		& Body-Boundary 2: & text/plain (deviceId) \\
		& Body-Boundary 3: & application/octet-stream (Video) \\
	\textbf{Response:} & Header: & HTTP/1.1 200 OK \\
		& Body: & <empty> \\
	\textbf{Zeitpunkt:} & \multicolumn{2}{p{10cm}}{Bei laufender Aufnahme, falls der Videostream aktiviert ist} \\ 
	\textbf{Interval:} & \multicolumn{2}{p{10cm}}{einstellbar} \\
\hline
\hline
	\textbf{URL:} & \multicolumn{2}{p{10cm}}{/api/racesituation/stage/\{stageId\}} \\
	\textbf{Request:} & Method: & POST \\
		& Body: & JSON-String 'RaceSituation'\\	
	\textbf{Response:} & Header: & HTTP/1.1 200 OK \\
		& Body: & <empty> \\	
	\textbf{Bemerkung:} & \multicolumn{2}{p{10cm}}{Der RadioTourSpeaker sendet (unabhängig von den TourLive Aufnahmegeräten) periodisch die Rennsituation an den Server} \\
\hline
\hline
	\textbf{URL:} & \multicolumn{2}{p{10cm}}{/api/getstageinfo/\{deviceId\}} \\
	\textbf{Request:} & Method: & GET \\
		& Body: & <empty>\\	
	\textbf{Response:} & Header: & HTTP/1.1 200 OK \\
		& Content-Type: & application/json \\
		& Body: & Custom JSON-String \\	
	\textbf{Bemerkung:} & \multicolumn{2}{p{10cm}}{Im Body der Response wird eine eigens kreierte HashMap ausgeliefert, darin sind aktuelle Etappeninformationen für das angegebene Gerät. Diese Informationen werden für den Rennbegleiter auf dem Gerät dargestellt} \\
\hline
\hline 
\caption{Schnittstellen TourLive Server - Aufnahmegerät}
\end{longtable} }

\section{TourLive Public API}
\label{sec:tourlivepublicapi}
Die Renn- und Etappeninformationen stehen für Drittentwickler zur Verfügung. Zu jeder Etappe können zusätzlich die ValueContainer und Bilddaten abgefragt werden.

{\renewcommand{\arraystretch}{1}
\begin{longtable}{ p{2.5cm} p{3.5cm} p{6cm}}
	\textbf{URL:} & \multicolumn{2}{l}{/public/stages} \\
	\textbf{Request:} & Method: & GET \\
		& Body: & <empty>\\
	\textbf{Response:} &  Header: & HTTP/1.1 200 OK \\
		& Content-Type: & application/json \\
		& Body: & JSON-Array 'Stages'\\
	\textbf{Bemerkung:} & \multicolumn{2}{p{10cm}}{Alle sichtbaren Etappen} \\
\hline
\hline    
	\textbf{URL:} & \multicolumn{2}{l}{/public/stage/\{stageId\}/valuecontainer} \\
	\textbf{Request:} & Method: & GET \\
		& Body: & <empty>\\
	\textbf{Response:} &  Header: & HTTP/1.1 200 OK \\
		& Content-Type: & application/json \\
		& Body: & JSON-Array 'ValueContainers'\\
	\textbf{Bemerkung:} & \multicolumn{2}{p{10cm}}{Alle ValueContainers einer Etappe} \\
\hline
\hline
	\textbf{URL:} & \multicolumn{2}{l}{/public/stage/\{stageId\}/imagedata} \\
	\textbf{Request:} & Method: & GET \\
		& Body: & <empty>\\
	\textbf{Response:} &  Header: & HTTP/1.1 200 OK \\
		& Content-Type: & application/json \\
		& Body: & JSON-Array 'ImageData'\\
	\textbf{Bemerkung:} & \multicolumn{2}{p{10cm}}{Alle ImageData Objekte zu dieser Etappe, nicht aber die eigentlichen Bilder. Die Bilder können aber mit dem Feld \textit{imageLocation} entweder direkt verlinkt oder heruntergeladen werden} \\
\hline
\hline
	\textbf{URL:} & \multicolumn{2}{l}{/public/stage/\{stageId\}/videodata} \\
	\textbf{Request:} & Method: & GET \\
		& Body: & <empty>\\
	\textbf{Response:} &  Header: & HTTP/1.1 200 OK \\
		& Content-Type: & application/json \\
		& Body: & JSON-Array 'VideoData'\\
	\textbf{Bemerkung:} & \multicolumn{2}{p{10cm}}{Alle VideoData Objekte zu dieser Etappe, nicht aber die eigentlichen Videosequenzen. Die Videos können aber mit dem Feld \textit{videoLocation} entweder direkt verlinkt oder heruntergeladen werden} \\
\hline
\hline
	\textbf{URL:} & \multicolumn{2}{l}{/public/stage/\{stageId\}/marchtableitem} \\
	\textbf{Request:} & Method: & GET \\
		& Body: & <empty>\\
	\textbf{Response:} &  Header: & HTTP/1.1 200 OK \\
		& Content-Type: & application/json \\
		& Body: & JSON-Array 'MarchTableItem'\\
	\textbf{Bemerkung:} & \multicolumn{2}{p{10cm}}{Die Marschtabelle ist aufgeteilt in Einheiten. Jede Reihe bedeutet eine Einheit. Pro Etappe können alle Marschtabelleneinheiten abgerufen werden.} \\
\hline
\hline
	\textbf{URL:} & \multicolumn{2}{l}{/public/stage/\{stageId\}/riders} \\
	\textbf{Request:} & Method: & GET \\
		& Body: & <empty>\\
	\textbf{Response:} &  Header: & HTTP/1.1 200 OK \\
		& Content-Type: & application/json \\
		& Body: & JSON-Array 'Riders'\\
	\textbf{Bemerkung:} & \multicolumn{2}{p{10cm}}{Sämtliche Fahrer, welche dieser Etappe zugeordnet sind} \\
\hline
\hline
\caption{TourLive Public API}
\end{longtable}}

\section{Device Management Portal Schnittstelle}
{\renewcommand{\arraystretch}{1}
    \begin{longtable}{ p{2.5cm} p{3.5cm} p{6cm}}
	\textbf{URL:} & \multicolumn{2}{l}{/devmgmtsrv/api/postdevicemanagementcontainer} \\
	\textbf{Request:} & Method: & POST \\
		& Content-Type: & application/json \\
		& Body: & JSON-String 'DeviceManagementContainer'\\
	\textbf{Response:} &  Header: & HTTP/1.1 200 OK \\
		& Body: & <empty>	\\
	\textbf{Eigenschaft:} & \multicolumn{2}{p{10cm}}{Wird einmalig beim App Start ausgeführt sowie bei lokalen Änderungen in den Einstellungen} \\
	\textbf{Interval:} & \multicolumn{2}{p{10cm}}{unregelmässig} \\
\hline
\hline    
	\textbf{URL:} & \multicolumn{2}{l}{/devmgmtsrv/api/getdevicemanagementcontainer} \\
	\textbf{Request:} & Content-Type: & application/json \\
		& Method: & POST \\
		& Body: & JSON-String 'StatusData' \\
	\textbf{Response:} & Method: & POST \\
		& Content-Type: & application/json \\
		& Body: & JSON-String 'DeviceManagementContainer' \\
	\textbf{Eigenschaft:} & \multicolumn{2}{p{10cm}}{Wird regelmässig als Service ausgeführt.} \\ 
	\textbf{Interval:} & \multicolumn{2}{p{10cm}}{regelmässig - alle 60 Sekunden} \\
\hline
\hline    
	\textbf{URL:} & \multicolumn{2}{p{10cm}}{/devmgmtsrv/api/postlog} \\
	\textbf{Request:} & Method: & POST \\
		& Content-Type: & application/json \\
		& Body: & JSON-String 'TourLiveLog' \\
	\textbf{Response:} &  Header: & HTTP/1.1 200 OK \\
		& Body: & <empty>	\\
	\textbf{Eigenschaft:} &  \multicolumn{2}{p{10cm}}{Wird regelmässig als Service ausgeführt.}\\ 
	\textbf{Interval:} &  \multicolumn{2}{p{10cm}}{regelmässig - alle 60 Sekunden}\\
\hline
\hline 
	\textbf{URL:} & \multicolumn{2}{p{10cm}}{/devmgmtsrv/api/getMsg/\{deviceId\}} \\
	\textbf{Request:} & Method: & GET \\
		& Body: & <empty> \\
	\textbf{Response:} & Method: & POST \\
		& Content-Type: & application/json \\
		& Body: & JSON-String 'Message' \\
	\textbf{Eigenschaft:} & \multicolumn{2}{p{10cm}}{ Wenn das 'messageAvailable'-Flag gesetzt ist in einem empfangenen DeviceManagementContainer.} \\
	\textbf{Interval:} & \multicolumn{2}{p{10cm}}{unregelmässig}\\
\hline
\hline 

\caption{Schnittstellen DeviceManagement Portal Übersicht}
\end{longtable} }

\pagebreak
\subsection{Status posten und Einstellungen erhalten}

Um den aktuellen Status des Aufnahmesystems dem Server mitzuteilen kann diese Methode aufgerufen werden. Als Antwort erhält man die im Moment aktuellen Einstellungen.

{\bf URL: }http://tlng.cnlab.ch/devmgmtsrv/api/getdevicemanagementcontainer 

\subsubsection{Status JSON}

Das JSON eines Status Posts sieht folgendermassen aus:

\begin{figure}[H]
	\centering
	\lstinputlisting[language=json]{jsonfiles/status.json}
	\caption{Status JSON}
\end{figure}


\pagebreak
\subsubsection{Einstellungen JSON}

Die Antwort die man bei dieser Anfrage erhält ist die folgende:

\begin{figure}[H]
	\centering
	\lstinputlisting[language=json]{jsonfiles/settings.json}
	\caption{Einstellungen JSON}
\end{figure}


\subsection{Log posten}

Um neue Logeinträge dem Server zu übermitteln kann diese Methode benutzt werden.

{\bf URL: }http://tlng.cnlab.ch/devmgmtsrv/api/postlog

\subsubsection{Log JSON}

Das JSON eines Log Posts sieht folgendermassen aus:

\begin{figure}[H]
	\centering
	\lstinputlisting[language=json]{jsonfiles/log.json}
	\caption{Log JSON}
\end{figure}

\subsection{Nachricht abholen}

Sofern eine neue Nachricht für das Gerät vorhanden ist kann sie über diese Methoden abgeholt werden.

{\bf URL: }http://tlng.cnlab.ch/devmgmtsrv/api/getmsg/{deviceId}

\subsubsection{Nachrichten JSON}
Das JSON für die Nachrichten ist das folgende:
\begin{figure}[H]
	\centering
	\lstinputlisting[language=json]{jsonfiles/message.json}
	\caption{Nachrichten JSON}
\end{figure}


