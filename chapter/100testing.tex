\section{Testing}
Das Testing des Gesamtsystems und der einzelnen Komponenten wurde mit verschiedenen Testverfahren abgedeckt. Funktionen und Algorithmen innerhalb der einzelnen Komponenten wurden mit Unit Tests (Kapitel \ref{sec:unittests}) getestet. Systemtests (Kapitel \ref{sec:systemtests}) wurden durchgeführt um das Zusammenspiel der verschiedenen Komponenten und ihre Funktionen als Ganzes zu testen. Auf Usability Tests wurde absichtlich verzichtet, da sich die Komplexität der grafischen Oberflächen in Grenzen hält. 

\subsection{Systemtests}
\label{sec:systemtests}

\subsubsection{Übersicht Systemtests}
\begin{longtable}{p{0.3cm}|p{1.8cm}|p{9.3cm}}
& &  \\ [-1.5ex]
\textbf{\#} & \textbf{Datum} & \textbf{Systemtest} \\ [1ex] \hline & &  \\ [-1.5ex]
1 & 25.05.2013 & Rapperswiler Rundfahrt \\ [1ex] \hline & &  \\ [-1.5ex]
2 & 01.06.2013 & Langzeittest über Nacht \\ [1ex] \hline & &  \\ [-1.5ex]
3 & 02.06.2013 & Basel-Zürich \\ [1ex] \hline & &  \\ [-1.5ex]
4 & 06.06.2013 & Gippingen \\ [1ex] \hline & &  \\ [-1.5ex]
5 & 09.06.2013 & Tour de Suisse \\ [1ex]

\caption{Übersicht aller durchgeführten Systemtests}
\end{longtable} 

Nachfolgend werden sämtlich durchgeführten Systemtests beschrieben und ausgewertet. Im Kapitel Resultate werden die Erkenntnisse aus den Systemtests zusammengefasst und wie folgt klassifiziert:
\begin{itemize} [noitemsep,topsep=0pt]
	\item[+] \textit{positive Erkenntnis}
	\item[-] \textit{negative Erkenntnis} 
\end{itemize}

Zu jeder negativen Erkenntnis wird eine Massnahme vorgeschlagen. Massnahmen die nicht mehr im Rahmen dieser Arbeit umgesetzt werden können werden mit einem  \textsuperscript{*} markiert.

\subsubsection{\#1 - Systemtest Rapperswiler Rundfahrt}
Ein erster Praxistest um das Zusammenspiel aller drei Komponenten zu testen. Auf einer Rundfahrt durch Rapperswil mit einem einzelnen Auto konnte der eine oder andere Fehler eruiert werden. Folgende Geräte wurden zu Testzwecken eingesetzt:
\begin{itemize}
	\item{Google Galaxy Nexus}
	\item{Samsung Galaxy S3}
\end{itemize}

\paragraph*{Resultate}

Folgende Erkenntnisse wurden aus dem Systemtest gewonnen:
\begin{itemize}[noitemsep,topsep=0pt]
	\item[-] App: Fehlberechnung bei der Distanz
	\item[-] App: Fehlberechnung bei der Renndauer
	\item[-] App: Reproduzierbare Abstürze (NullPointerExceptions,...)
	\item[-] App: NullPointerException wenn Gerät im Landscape-Modus
	\item[-] TourLive Server: DataIntegrityDataViolation Exception
	\item[-] TourLive Server: Inkonsistenzen I18N
	\item[-] TourLive Server: durchschnittliche Geschwindigkeit fehlt
	\item[-] Fehlerhafte Werte im übertragenen ValueContainer
	\item[+] Funktionierendes Zusammenspiel der Komponenten
\end{itemize}
Sämtliche negativen Erkenntnisse wurden behoben und werden an dieser Stelle daher nicht weiter erläutert.

\subsubsection{\#2 - Systemtest Langzeittest über Nacht}
Aufgrund wiederkehrender Abstürze der TourLive App bei laufender Aufnahme wurde über Nacht ein Langzeittest durchgeführt. Beide Geräte wurden einer 8 stündigen Etappe zugewiesen und sollten über Nacht Positionsdaten und Bilder übertragen. 
\begin{itemize} [noitemsep,topsep=0pt]
	\item{Google Galaxy Nexus}
	\item{Samsung Galaxy S3}
\end{itemize}

\paragraph*{Resultate}
Folgende Erkenntnisse wurden aus dem Systemtest gewonnen:
\begin{itemize}
\item[-] Samsung Galaxy S3: App ist nach rund 2 Stunden aufgrund eines SIG-9 FATAL ERROR's abgestürzt. Grund für solche Abstürze sind Schreib-/Lesekonflikte im Speicherbereich und treten auf wenn zwei Threads gleichzeitig eine gemeinsame Variabel lesen / schreiben. Aufgrund dieser Erkenntnis wurden die diversen AsyncTasks, die parallel ausgeführt werden, optimiert.
\item[+] Google Galaxy Nexus: Hat den Test erfolgreich bestanden. Im Log konnten keine Verdächtigen Einträge festgestellt werden.
\end{itemize}
Sämtliche negativen Erkenntnisse wurden behoben und werden daher an dieser Stelle nicht weiter erläutert.



\subsubsection{\#3 - Systemtest Basel - Zürich}
Während einer längeren Autofahrt wurde das System als Ganzes einem grundlegenden Test unterzogen. Bei beiden Geräten wurde als Streamingmodus die Aufnahme von Bildern konfiguriert.

\begin{itemize} [noitemsep,topsep=0pt]
	\item Google Galaxy Nexus [Orange SIM-Karte]
	\item Samsung Galaxy S3 [Swisscom SIM-Karte]
\end{itemize}

\paragraph*{Resultate}
Beide Geräte sendeten Bilder und Positionsdaten an den TourLive Server. Beim Google Galaxy Nexus mit der Orange SIM-Karte wurden aufgrund der schlechteren Netzabdeckung weniger Übertragungen registriert. Die TourLive Applikation auf dem Samsung Galaxy S3 musste 2x aufgrund eines Crashes (NullPointerException) neugestartet werden.

Sämtliche negativen Erkenntnisse wurden behoben und werden daher an dieser Stelle nicht weiter erläutert.


\subsubsection{\#4 - Systemtest Radsporttage Gippingen}
\label{sec:testberichtgippingen}
Im Rahmen der Radsporttage Gippingen konnte das Gesamtsystem ein erstes Mal praxisnah an einem Radrennen getestet werden. Für den Testlauf an den Radsporttagen in Gippingen wurden drei verschiedene Androidgeräte mit der aktuellsten Version der TourLive App ausgerüstet. Auf dem TourLive Server wurde ein Rennen und eine Etappe erstellt und die Fahrerliste sowie die Marschtabelle importiert. Im Einsatz waren folgende Gerätetypen:
\begin{itemize} [noitemsep,topsep=0pt]
	\item Google Galaxy Nexus [Orange SIM-Karte] im VIP-Auto
	\item Samsung Galaxy S3 [Swisscom SIM-Karte] im VIP-Auto
	\item HTC One [Sunrise SIM-Karte] im Feld
	\item Samsung Galaxy S3 mini [Swisscom SIM-Karte] im TourSpeaker Auto
	\item Samsung Galaxy S3 [Orange SIM-Karte] an der Spitze
\end{itemize}

\paragraph*{Resultate}
Nach wenigen Minuten konnte kein Kontakt mehr zum HTC One Aufnahmegerät hergestellt werden. Das Gerät konnte durch die Notfallwiederherstellung per SMS neu gestartet werden, jedoch war nach kurzem Kontakt die Verbindung zum Gerät wieder abgebrochen. Die erste Analyse ergab, dass das Gerät stark überhitzt war und sich deshalb selbst ausschaltete. Ein weiteres Gerät versuchte sich mit dem naheliegenden Deutschen Mobilfunknetz zu verbinden, was aber daran scheiterte, dass das Datenroaming deaktiviert war. Das dritte Gerät sowie die zusätzlichen Testgeräten hielten den Test durch, obwohl alle Geräte extrem warm geworden sind. Die eingelieferten Daten wurden vom TourLive Server empfangen und verarbeitet. Die Kommunikation zwischen Aufnahmegeräten und Server verlief problemlos. Zu Bemerken ist, dass 2 der eingesetzten Geräte, mit denen es auch am meisten Probleme gab, zuvor nicht getestet werden konnten. Zusammenfassend können folgende Erkenntnisse aus diesem Test mitgenommen werden:

\begin{itemize} [noitemsep,topsep=0pt]
\item[-] Grundsätzliche Probleme (Exceptions) aufgrund Programmierfehler mit den beiden Gerätetypen HTC One und Samsung Galaxy S3 mini. Excetions die nachvollzogen werden konnten wurden behoben.
\item[-] Starke Erhitzung der Geräte aufgrund direkter Sonneneinstrahlung was zu starken Performanceeinbussen, beispielsweise bei der Bedienung des GUI's führt. Massnahme:\textsuperscript{*} Direkte Sonneneinstrahlung wenn möglich vermeiden. Die Intervalle zur Übertragung von Daten heruntersetzen um die Auslastung der Geräte zu vermindern.
\item[-] Performanceeinbussen bei wachsender Grösse des lokalen Caches (ORMlite Datenbank). Massnahme:\textsuperscript{*} sämtliche Daten die bereits übertragen wurden, müssen nicht mehr zwingend lokal gespeichert werden. Dies würde die Grösse der lokalen Datenbank vermindern. Eine andere Möglichkeit wäre der Einsatz eines Alternativproduktes.
\item[-] Optimierungsmöglichkeiten bei der Darstellung der Geräte im Device Management Portal aufgrund der praktischen Anwendung. Hierbei handelt es sich primär um die Sortierung entsprechender Listen, was bereits umgesetzt wurde.
\item[+] Wichtigkeit der Fernwartungsmöglichkeiten erkannt und mit Erfolg eingesetzt.
\item[+] Zuverlässig funktionierende Aufnahme mit dem Gerätetyp Samsung Galaxy S3
\item[+] Weniger als 1 km Abweichung am Etappenende bei der Berechnung der Tour-Distanz im Vergleich zum bestehenden TourLive System
\end{itemize}

\subsubsection{\#5 - Systemtest Tour de Suisse - 2. Etappe}
Ein letzter Systemtest konnte an der 2. Etappe der Tour de Suisse durchgeführt werden. Im Einsatz waren folgende Gerätetypen:
\begin{itemize} [noitemsep,topsep=0pt]
	\item HTC One [Sunrise SIM-Karte]
	\item Samsung Galaxy S3 mini [Swisscom SIM-Karte]
	\item Samsung Galaxy S3 [Orange SIM-Karte]
\end{itemize}
\paragraph*{Resultate}
\begin{itemize}
	\item[-] Absturz der TourLive Android Applikation auf dem Samsung Galaxy S3 während laufender Aufnahme von Positionsdaten beim Ändern des Streamingmodus per Device Management Server. Wahrscheinlich gab es Probleme beim Neustart des RecordingService. Als Massnahme\textsuperscript{*} wird empfohlen das Ändern von Einstellungen während laufender Aufnahme experimentell zu testen um das genaue Problem zu eruieren. 
	\item[-] Bei schlechten Lichtverhältnissen wird nicht erkannt wenn der Aufnahme-Button aufgrund eines falschen Aufnahmemodus deaktiviert ist. Als Massnahme wurde die Farbe des deaktivierten Start-Button in rot geändert.
	\item[-] Probleme gab es der Berechnung der Abstandsentwicklung auf dem TourLive Server. Grund waren falsche Rennkilometerkorrekturen auf den Aufnahmegeräten. Massnahme:\textsuperscript{*} Der Algorithmus zur Berechnung der Abstandsentwicklung auf dem TourLive Server sollte zur Berechnung der Abstandsentwicklung nicht die berechneten Distanzwerte der Aufnahmegeräte verwenden, sondern die Koordinaten im PositionDate Container.
	\item[+] Das Aufnahmegerät Samsung Galaxy S3 lief mit Ausnahme des Absturzes aufgrund Konfigurationsänderungen während der gesamten Etappe stabil. 
	\item[+] Das Senden von Nachrichten an die Aufnahmegeräte funktionierte
	\item[+] Die Rennkilometerkorrektur via Device Management Server funktionierte.
\end{itemize}

\subsubsection{Fazit der Systemtests}
\label{sec:fazitsystemtests}

Aufgrund der Systemtests und spontaner Ideen während der Entwicklung und dem Testing wurden folgende Verbesserungsmöglichkeiten aller drei Komponenten zusammengestellt.
\paragraph{TourLive Server}
%TODO flo erweitern
\begin{itemize}
 \item Der Algorithmus für die Abstandsentwicklung basiert aktuell auf Basis der berechneten Etappendistanzen der Aufnahmegeräte. Anstelle dieser berechneten Distanzen könnte die Abstandsentwicklung auch auf Basis der Koordinaten im ValueContainer berechnet werden. Dies würde verhindern, dass allfällige Rechnungsfehler der Aufnahmegeräte nicht in die Abstandsentwicklung einflössen.
\end{itemize}

\paragraph{Android Aufnahmesystem}
\begin{itemize}
	\item Der Start / Stop des Aufnahmemodus (RecordingService) scheint sofern Bilder- oder Videostreaming aktiviert ist sporadisch Probleme zu bereiten. Dies zeigt sich vor allem wenn der Aufnahmemodus per Fernsteuerung (Device Management Server) gestartet / gestoppt wird. Die genaue Fehlerquelle konnte bisher nicht eruiert werden.
	\item Entwicklung eines eigene Thread-Pools zur Verwaltung und Abarbeitung der Threads (inbesondere der AsyncTasks). Tests haben ergeben, dass die Anzahl möglicher paralleler Threads von Gerät zu Gerät variiert und auch der Android interne Thread-Pool teilweise unterschiedlich arbeitet. 
	\item Die Verwendung von SQLite in Verbindung mit ORMlite führt bei grösseren Datenmengen zu erheblichen Performanceeinbussen. Die Datenmenge könnte reduziert werden, indem bereits übertragene Daten im lokalen Cache gelöscht werden oder indem ein alternatives, performanteres Produkt eingesetzt würde.
	\item Das Ausschalten des Bildschirms / \"Schliessen\" der Applikation führt beim Einsatz von Bilder- / Videostreaming sporadisch dazu, dass die Fragments innerhalb der MainActivity nicht mehr korrekt angezeigt werden. Eine exakte Fehlerquelle konnte bisher nicht eruiert werden. 
	\item Befindet sich die TourLive Applikation im Aufnahmemodus \"ferngesteuert\" und in einer laufenden Aufnahme, kann das Aufnahmegerät ohne Löschung der gesamten Einstellungen nicht mehr verwaltet werden. Dies liegt daran, dass die Einstellungen während einer laufenden Aufnahme gesperrt sind und die Aufnahme nicht gestoppt werden kann, da der Aufnahmemodus auf \"ferngesteuert\" gesetzt ist. 
	
\end{itemize}

\paragraph{Device Management Server}
\begin{itemize}
	\item another anything else
\end{itemize}
%TODO whats here

\subsection{Unit Tests}
\label{sec:unittests}
Für den TourLive Server existieren grundlegende JUnit Tests. Diese dienen in erster Linie dazu, die Serverkomponente zu testen und die Entwicklung von weiteren Tests beispielhaft zu dokumentieren.