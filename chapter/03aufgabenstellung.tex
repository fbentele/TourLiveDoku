\chapter*{Aufgabenstellung}

\begin{tabular}{ll}
	Studiengang: & Informatik (I) \\
	Institut: & ITA: Internet-Technologien und Anwendungen \\
	Gruppe: & Patrizia Heer, Simon Stäheli, Florian Bentele \\
	Betreuer: & Prof. Dr. Peter Heinzmann  \\
	Koreferent: & Prof. H. Huser, HSR \\
	Experte: & Dr. Th. Siegenthaler, CSI Consulting AG  \\
	Industriepartner: & Swiss Cycling / cnlab Software ag \\    
\end{tabular}
\\

\section*{Ausgangslage}
Das cnlab TourLive-System\footnote{Tourlive-System, \url{www.tourlive.ch}, aufgerufen am 30.04.2013} dient zur Renndatenerfassung an Sportanlässen. Es wird seit 2004 an der Tour de Suisse und bei den Rennsporttagen Gippingen eingesetzt. Die in Schiedsrichterfahrzeugen montierten mobilen Aufnahmesysteme (Nokia Mobiltelefone mit Symbian-Anwendung) erfassen die Position von den Spitzenfahrern, den Verfolgern und dem Feld. Ferner liefern die Aufnahmesysteme Live-Bilder aus Sicht dieser Schiedsrichterfahrzeuge. Mit Hilfe der RadioTour-Android Tablet Anwendung notiert der RadioTour-Speaker die Zeitabstände und Zusammensetzung von Fahrergruppen. All diese Daten werden auf dem cnlab TourLive-Server gesammelt, aufbereitet und für die Publikation auf Webservern zu Verfügung gestellt. 
Swiss Cycling möchte die TourLive-Anwendung nun auch kleineren Rennveranstaltern zu Verfügung stellen können. In diesem Zusammenhang sollen auch die Aufnahmesysteme auf Android portiert und in der Funktionalität optimiert werden.

\section*{Ziel}
Das überarbeitete TourLive-System soll als Gesamtpaket für Rennveranstalter zur Verfügung stehen. Die Rennveranstalter sollen ihre Anlässe zusammen mit den Renninformationen auf der Webanwendungen (Datenerfassungs- und Aufbereitungssystem) präsentieren können. Die Aufnahmesysteme sollen auf Android Smartphones funktionieren und sie sollen über eine Management Anwendung überwacht und konfiguriert werden können.
\\

\section*{Teilaufgaben}
\begin{itemize}
	\item Analyse
	\begin{itemize}
		\item Detailliertes Studium des aktuellen TourLive-Systems (Symbian Aufnahmesysteme und Webanwendung)
		\item Studium verschiedener Webseiten zu Radrennen, Erstellung einer Übersicht und Beurteilung der verschiedenen Elemente auf solchen Webseiten
		\item Festlegung der Funktionen der Aufnahme- und Darstellungssysteme
	\end{itemize}
	\item Entwicklung Android Aufnahmesysteme
	\begin{itemize}
		\item Positions- und Bildaufnahmen
		\item Alarming-Funktionen
		\item Logging-Funktionen
		\item Kommunikation mit verschiedenen Datenservern
	\end{itemize}
	\item Entwicklung Webanwendung (Webserver für Radrennen, bei welchen mit TourLive und RadioTour Renndaten aufgezeichnet und dargestellt werden)
	\begin{itemize}
		\item Funktionsblöcke (Zeitabstände, Fahrergruppen, Bilder, LiveTicker, Kartendarstellung, Höhenprofil, Ranglisten, Marschtabellen, Fahrerinformationen, Abstände von Aufnahmesystemen, Distanz zum Werbe- und Renntross, Zeit bis dieser eine bestimmte Stelle passiert, …)
		\item Kommunikationsschnittstelle mit alten und neuen TourLive-Aufnahmesystemen
		\item Kommunikationsschnittstelle mit RadioTour-Anwendung
		\item Schnittstelle zu Zeiterfassungssystemen (Matsport)
		\item Rennadministration (Fahrerlisten)
	\end{itemize}
	\item Testeinsatz am 50. GP des Kantons Aargau vom Donnerstag, 6. Juni 2013 im Rahmen der Radsporttage Gippingen 2013
\end{itemize}
