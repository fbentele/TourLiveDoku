\subsection{Florian Bentele}
Da es sich für mich bei dieser Bachelorarbeit um eine Art Fortsetzungsarbeit handelt, hatte ich bereits ein gutes Verständnis für das Aufgabenumfeld. Dennoch war die Aufgabenstellung, eine Webapplikation zu entwickeln, eine neue Herausforderung.

In den ersten Wochen musste ich sehr viel Zeit in das Erlernen des mir unbekannten Spring Framework investieren. Bei Fragen konnte ich wohl auf ein fundiertes Wissen bei Mitstudenten zurückgreifen, dennoch blieb der Einstieg eher schwierig, da mir die verwendeten Konzepte nicht geläufig waren. Erst nachdem der erste Prototyp erfolgreich getestet wurde, bekam ich das Gefühl, auf das richtige Framework gesetzt zu haben.

Die Aufgabenteilung im Team war im Ansatz gegeben. Unsere Aufgabe bestand darin, die Schnittstellen zu den Systemen zu definieren und dann die Systemkomponenten umzusetzen. Dies fordert eine gewisse Flexibilität, da manche Probleme erst während der Entwicklung auftauchen und die Schnittstellen danach angepasst werden müssen. Weiter war auch nicht immer klar, auf welcher Komponente ein Task ausgeführt werden soll. Die Drehung von aufgenommen Bildern sei an dieser Stelle als Beispiel erwähnt. Gemeinsam haben wir jedoch die Probleme besprochen und immer eine geeignete Lösung gefunden.

Ich habe in diesem Projekt sehr viel über die Entwicklung von Webapplikationen gelernt. Mir ist im besonderen die Wichtigkeit für den Einsatz eines Frameworks klar geworden. Für die professionelle Entwicklung ist die Verwendung eines Frameworks unabdingbar. Persönlich bleiben mir besonders die Erlebnisse am Testlauf an den Radsporttagen in Gippingen in guter Erinnerung.
