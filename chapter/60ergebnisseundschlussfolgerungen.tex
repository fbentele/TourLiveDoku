\chapter{Ergebnisse und Schlussfolgerungen}

Alle Komponenten dieser Arbeit wurden gemäss den Anforderungen erarbeitet und getestet. In den folgenden Abschnitten werden die Schlussfolgerungen erläutert und einen Ausblick über die weitere Entwicklung und den möglichen Einsatz an Radrennen gegeben.

\section{Endprodukt}
Die zu Beginn der Arbeit definierte Spezifikation sämtlicher Systeme konnten umgesetzt  und an den Radsporttagen in Gippingen einem Test unterzogen werden. Ein ausführlicher Testbericht befindet sich im Anhang \ref{sec:testberichtgippingen}. Die drei Komponenten, TourLive Server, die Aufnahmegeräte und der Device Management Server arbeiten wie erwartet zusammen.

\section{Ausblick}
Nach den positiven Erfahrungen an den Radsporttagen in Gippingen konnte das System auch an der Tour de Suisse getestet werden. In einem zweiten Live Test wurde der Einsatz parallel zum bestehenden System von der cnlab Software AG durchgeführt. Allfällige Verbesserungsmöglichkeiten und Ideen zur Weiterentwicklung von TourLive wurden im Kapitel \ref{sec:fazitsystemtests} im Anhang zusammengefasst. Allfällige Anpassungen am System werden durch die cnlab Software AG vorgenommen.

Für die Dauer der Entwicklung wurde der TourLive Server auf einem virtuellen Server betrieben. Beim Einsatz an einem Radrennen wie der Tour de Suisse wird empfohlen,  die Dienste auf mehrere Server aufzuteilen. Weiter hat sich herausgestellt, dass sich nicht alle Androidgeräte gleich gut als Aufnahmegeräte eignen. Während der Entwicklung wurde primär mit dem Gerät Samsung Galaxy Nexus\footnote{Samsung Galaxy Nexus, \url{http://www.gsmarena.com/samsung_galaxy_nexus_i9250-4219.php} aufgerufen am 12.06.2013} gearbeitet, welches auch die besten Testresultate erzielte. Ebenfalls zu empfehlen, und auch etwas aktuellere bezüglich Hardware, ist das Samsung Galaxy S3\footnote{Samsung Galaxy S3, \url{http://www.samsung.com/global/galaxys3/}, aufgerufen am 02.06.2013}.
