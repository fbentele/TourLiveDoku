\chapter{Ergebnisse und Schlussfolgerungen}

Alle Komponenten dieser Arbeit wurden gemäss den Anforderungen erarbeitet und getestet. In den folgenden Abschnitten werden die Schlussfolgerungen erläutert und einen Ausblick über die weitere Entwicklung und den möglichen Einsatz an Radrennen gegeben.

\section{Endprodukt}
Die zu Beginn der Arbeit definierte Spezifikation sämtlicher Systeme konnten umgesetzt werden und an den Radsporttagen in Gippingen einem Test unterzogen werden, ein ausführlicher Testbericht befindet sich im Anhang \ref{sec:testberichtgippingen}. Die drei Komponenten, TourLive Server, die Aufnahmegeräte und der Device Management Server arbeiten wie erwartet zusammen.
\\

\section{Ausblick}
Nach den positiven Erfahrungen an den Radsporttagen in Gippingen kann das System auch an der Tour de Suisse getestet werden. Dieser zweite Livetest liegt aber ausserhalb des Umfangs dieser Arbeit. Daher wird der Einsatz parallel zum bestehenden System von der cnlab AG durchgeführt. Allfällige Anpassungen am System werden dann ebenfalls durch die cnlab AG vorgenommen.
\\

Für die Dauer der Entwicklung wurde der TourLive Server auf einem virtuellen Server betrieben. Beim Einsatz an einem Radrennen wie der Tour de Suisse wird empfohlen die Dienste auf mehrere Server aufzuteilen. Weiter hat sich herausgestellt, dass sich nicht alle Androidgeräte gleich gut für die Aufzeichnungen eigenen. Der empfohlene Gerätetyp ist das Samsung Galaxy S3\footnote{Samsung Galaxy S3, \url{http://www.samsung.com/global/galaxys3/}, aufgerufen am 02.06.2013}
%TODO stimmt das mit dem Galaxy S3?
