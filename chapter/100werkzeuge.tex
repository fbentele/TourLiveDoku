\section{Werzeuge und Entwicklungsumgebung}
\label{sec:wekzeugeundentwicklungsumgebung}
Für die drei Teilprojekte TourLive Server, Device Management Server und Aufnahmesystem wurde jeweils die Java Programmiersprache verwendet. Dennoch gibt es Unterschiede bei den Entwicklungsumgebungen, diese sind im folgenden Abschnitt erläutert.

\subsection{TourLive Server}
\begin{itemize}
\item Entwicklungsumgebung: Spring Tool Suite (STS), basierend auf Eclipse, \url{http://www.springsource.org/sts}
\item Framework: Spring MVC, Java Webframework, \url{http://www.springsource.org/}
\item Mockups: Balsamiq Mockup für erste Entwürfe, \url{http://www.balsamiq.com/}
\item Versionierungssystem: git, \url{http://git-scm.com/}
\end{itemize}

\subsubsection{Installation und Deployment}
\label{sec:installationdeploymenttourliveserver}
Um das Projekt zu builden und auf einem Webserver zu betreiben sind folgende Schritte notwendig.

\begin{itemize}
\item Projekt aus GitHub klonen oder komprimierte zip Datei von der CD entpacken
\item In der pom.xml Datei ganz unten ein Profil für die Entwicklungsumgebung erstellen und vergewissern, dass der Datenbankserver läuft
\item Projekt mit Maven builden:
\end{itemize}

\begin{lstlisting}[language=Bash, caption=Build und Test mit Maven]
# bash oder andere Shell starten und
# ins Projektverzeichnis wechseln
~ $> cd RadioTourWebsite

# Dependencies herunterladen und Tests starten
RadioTourWebsite $> mvn install -P meinprofil

# Projekt kompilieren und als deployable war exportieren
RadioTourWebsite $> mvn package -P meinprofil

# Das War File liegt dann im Verzeichnis
RadioTourWebsite $> ~/RadioTourWebsite/target/ba-1.0.0-BUILD-SNAPSHOT.war

\end{lstlisting}

\begin{itemize}
\item Die \textit{war} Datei kann nun auf verschiedene Arten auf dem Tomcat deployed werden, am einfachsten ist das Autodeployment von Tomcat
\end{itemize}
\begin{lstlisting}[language=Bash, caption=Deployment auf Tomcat]
# war Datei aus dem target Ordner in den Tomcat webapps
# Ordner kopieren
RadioTourWebsite $> cp target/ba-1.0.0-BUILD-SNAPSHOT.war /var/lib/tomcat7/webapps/ROOT.war

# Bemerkung: Der Pfad zum webapps Ordner kann sich je
# nach Plattform unterscheiden. Wird die Datei in
# ROOT.war umbenannt so wird die Applikation auf dem
# Domainroot (http://tlng.cnlab.ch/) deployed.
# Danach Tomcat neu starten
RadioTourWebsite $> /etc/init.d/tomcat7 restart

\end{lstlisting}

Sämtliche Videos und Bilder werden nicht direkt über die Webapplikation ausgeliefert, sondern über einen konfigurierbaren Pfad (im Profil ganz unten im pom.xml) abgelegt und von dort aus zur Verfügung gestellt. Der empfohlene Webserver für den Betrieb ist ein Tomcat für die Auslieferung der Bilder und Videodateien kann ein anderer Server verwendet werden, um den Hauptwebserver zu entlasten. Dabei kann aber auch ein Tomcat verwendet werden, die Aussage, dass der Tomcat Server statische Inhalte langsamer ausliefert, wirkt sich gemäss Mark Thomas im Betrieb kaum auf die Performance aus \cite{thomas2010}.

\subsection{Device Management Server}
\begin{itemize}
\item Entwicklungsumgebung: Spring Tool Suite (STS), basierend auf Eclipse, \url{http://www.springsource.org/sts}
\item Framework: Spring MVC, Java Webframework, \url{http://www.springsource.org/}
\item Mockups: Evolus Pencil für erste Entwürfe, \url{http://pencil.evolus.vn/}
\item Versionierungssystem: git, \url{http://git-scm.com/}
\end{itemize}

\subsubsection{Installation und Deployment}
Um den Device Management Server zu builden, können die gleichen Schritte ausgeführt werden wie beim TourLive Server. Was alles gemacht werden muss ist im Kapitel \ref{sec:installationdeploymenttourliveserver} beschrieben.

\subsection{Aufnahmesystem}
\begin{itemize}
\item Entwicklungsumgebung: Android Development Tools (ADT), basierend auf Eclipse, \url{https://dl-ssl.google.com/android/eclipse/}
\item Mockups: Evolus Pencil für erste Entwürfe, \url{http://pencil.evolus.vn/}
\item Versionierungssystem: git, \url{http://git-scm.com/}
\end{itemize}

\paragraph{Android Version}
Eine Anwendung wird f"{u}r eine spezifische Android Version entwickelt und getestet. Somit kann garantiert werden, dass das Verhalten der Anwendung immer gleich ist. In dieser Arbeit ist dies die Version 4.0 mit dem Versionsnamen Ice Cream Sandwich\footnote{Android Ice Cream Sandwich, \url{http://www.android.com/about/ice-cream-sandwich/}, aufgerufen am 29.04.2013}.
Die Entwicklung auf einer Version schliesst jedoch nicht aus, dass die Anwendung in neueren Versionen nicht mehr lauff"{a}hig ist.