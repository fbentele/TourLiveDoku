\chapter*{Abstract}
Das cnlab TourLive System, kurz TourLive, ermöglicht die Aufzeichnung von Positions-, Bild- und Videodaten aus dem Fahrerfeld bei Radrennen. Die mittels Nokia Symbian Geräten gesammelten Daten werden an den TourLive Server übertragen und dort Radrennsportinteressierten präsentiert.

%Um den aktuellen Stand von Radrennen zu erfassen und diese Daten zu veröffentlichen, konnte bisher das TourLive System von der cnlab AG eingesetzt werden.

%Nokia Handys wurden in Begleitfahrzeugen an der Frontscheibe eingebaut und haben die aktuelle Position sowie Bilder aufgenommen. Diese Daten wurden dann an den TourLive Server geschickt, welcher die Informationen verarbeitet und als Webseite für Radsport interessierte präsentiert.

Im Rahmen dieser Bachelorarbeit wurde das über 10 jährige System analysiert, an die aktuellen Bedürfnisse angepasst und mit Hilfe moderner Technologien umgesetzt. Das TourLive System umfasst neu drei komplett voneinander getrennte Komponenten. Diese sind zum ersten die Aufnahmegeräte, welche durch Android Smartphones ersetzt wurden, zum zweiten der eigentliche TourLive Server (Java Spring Webservice) und als drittes das Geräteverwaltungsportal (Java Spring Webservice). 


Das Geräteverwaltungsportal ermöglicht die vollumfängliche Fernverwaltung und Überwachung der Aufnahmesysteme, welche bisher nur in einem rudimentären Rahmen möglich war. Die Administrationsoberfläche des TourLive Servers ermöglicht eine komfortable Renn- und Etappenverwaltung. Neben den bisherigen Elementen der Webseite ist es neu auch möglich etappenspezifische Werbebanner zu platzieren. Unter Verwendung moderner Webtechnologien kann die korrekte  Darstellung der Webseite auf mobilen Endgeräten sichergestellt werden.


Neu ist auch die Art des Videostreams. Während bisher nur Einzelbilder übertragen und auf dem Server zu einem Videostream verarbeitet wurden, ermöglicht das neue System die Übertragung ganzer Videosequenzen. Die Kommunikation zwischen den verschiedenen Komponenten erfolgt über eine RESTful- JSON- Schnittstelle. 


Das Endprodukt beinhaltet die vom Industriepartner gewünschte Funktionalität und einen Prototypen, welcher anhand eines Probelaufs an den 50. Radsporttagen Gippingen Anfang Juni 2013 erfolgreich getestet wurde. Die Weiterentwicklung und der allfällige Einsatz an Radrennen in der Schweiz liegt in den Händen von cnlab Software AG und Swisscycling.
