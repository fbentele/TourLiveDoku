\chapter*{Abstract}
Um den aktuellen Stand von Radrennen zu erfassen und diese Daten zu veröffentlichen, konnte bisher das TourLive System von der cnlab AG eingesetzt werden. Nokia Handys wurden in Begleitfahrzeugen an der Frontscheibe eingebaut und haben die aktuelle Position sowie Bilder aufgenommen. Diese Daten wurden dann an den TourLive Server geschickt, welcher die Informationen verarbeitet und als Webseite für Radsport interessierte präsentiert.
\\
Im Rahmen dieser Bachelorarbeit wurde das über 10 jährige System überarbeitet und an die neuen Anforderungen angepasst. Die TourLive Anwendung fasst drei Systeme zusammen. Zum Einen die Aufnahmegeräte, welche durch moderne Android Smartphones ersetzt werden. Das zweite Element bildet der TourLive Server, welcher die Daten der Aufnahmegeräte entgegennimmt und darstellt. Die dritte Systemkomponente ist das Geräteverwaltungsportal in welchem die Aufnahmegeräte auch nach dem Start eines Rennens aus der Ferne (Remote) bedient und überwacht werden können.
\\
Das Endprodukt beinhaltet die vom Industriepartner gewünschte Funktionalität und einen Prototypen, welcher anhand eines Probelaufs erfolgreich getestet wurde. Die weitere Entwicklung und der allfällige Einsatz an Radrennen in der Schweiz liegt in den Händen des Industriepartners.