\chapter*{Abstract}
Das cnlab TourLive System, kurz TourLive, ermöglicht die Aufzeichnung von Positions-, Bild- und Videodaten aus dem Fahrerfeld bei Radrennen. Die mittels Nokia Symbian Geräten gesammelten Daten werden an den TourLive Server übertragen und dort Radrennsportinteressierten präsentiert.
\\

Im Rahmen dieser Bachelorarbeit wurde das über 10 jährige System analysiert, an die aktuellen Bedürfnisse angepasst und mit Hilfe moderner Technologien umgesetzt. Das TourLive System umfasst drei voneinander getrennte Komponenten. Diese sind zum ersten die Aufnahmegeräte, welche durch Android Smartphones ersetzt wurden, zum zweiten der eigentliche TourLive Server (Java Spring Webservice) und als drittes der Device Management Server (Java Spring Webservice). 
\\

Die neu entwickelte Geräteverwaltung ermöglicht die vollumfängliche Fernverwaltung und Überwachung der Aufnahmesysteme, welche bisher nur in einem rudimentären Rahmen möglich war. Die Administrationsoberfläche des TourLive Servers ermöglicht eine komfortable Renn- und Etappenverwaltung. Neben den bisherigen Elementen der Webseite ist es neu auch möglich etappenspezifische Werbebanner zu platzieren. Unter Verwendung moderner Webtechnologien kann die korrekte  Darstellung der Webseite auch auf mobilen Endgeräten sichergestellt werden.
\\

Neu ist auch die Art des Videostreams. Während bisher nur Einzelbilder übertragen und auf dem Server zu einem Videostream verarbeitet wurden, ermöglicht das neue System die effiziente Übertragung ganzer Videosequenzen. Die Kommunikation zwischen den verschiedenen Komponenten erfolgt über eine RESTful- JSON- Schnittstelle. 
\\

Der Prototyp des Systems wurde anhand eines Probelaufs an den 50. Radsporttagen Gippingen sowie an der zweiten Etappe der Tour de Suisse Anfang Juni 2013 erfolgreich getestet. Wichtige Erkenntnisse aus diesen beiden Feldtests wurden dokumentiert, konnten im Rahmen dieser Arbeit aber nur noch teilweise berücksichtigt werden. Die Weiterentwicklung und der allfällige Einsatz an Radrennen liegt in den Händen der cnlab Software AG und Swisscycling.
