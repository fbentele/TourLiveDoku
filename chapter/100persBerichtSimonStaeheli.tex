\subsection{Simon Stäheli}
Die grosse Attraktivität an dieser Bachelorarbeit bestand für mich aus dem engen Praxisbezug. Zu Radrennen hatte ich zuvor keinen grossen Bezug. So mussten erst einmal ein Grundverständnis für die Abläufe während eines solchen Rennens erarbeiten werden. 
\\

Die Aufgabenteilung dieses doch ziemlich umfangreichen Projektes war von Anfang an gegeben. Zusammen mit Patrizia Heer war ich für das Android Aufnahmesystem (Android Applikation) und denn Device Management Server (Java Spring Framework Werbservice) verantwortlich. Da ich mit keiner der beiden verwendeten Plattformen Praxiserfahrung vorzuweisen habe, bestanden die ersten Wochen primär aus dem Erarbeiten der  Grundkenntnisse dieser beiden Technologien. Interessant wurde es als die ersten Prototypen getestet werden konnten und die ersten Bilder und Positionsdaten vom Aufnahmegerät an den TourLive Server übertragen wurden.
\\

Das Highlight dieses Projektes war sicherlich der Feldtest an den Radsporttagen in Gippingen und dass wir die Aufnahmegeräte an die zweite Etappe der Tour de Suisse mitgeben durften. 


