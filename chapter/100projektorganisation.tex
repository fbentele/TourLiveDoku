\section{Projektmanagement}
Für die Projektplanung und -verwaltung wurde auf verschiedene Hilfsmittel zurückgegriffen. Zum Projektbeginn wurde ein Grobzeitplan erstellt und folgende Meilensteine definiert:

\begin{longtable}{r|p{7.3cm}|l}
& &  \\ [-1.5ex]
\textbf{\#} & \textbf{Beschreibung} & \textbf{Datum} \\ [1ex] \hline \hline & &  \\ [-1.5ex]
0 & Kickoff Meeting & 31.01.2013 \\ [1ex] \hline & &  \\ [-1.5ex]
1 & Requirements definiert & 15.03.2013 \\ [1ex] \hline & &  \\ [-1.5ex]
2 & Schnittstellen definiert & 22.03.2013 \\ [1ex] \hline & &  \\ [-1.5ex]
3 & Erster Prototyp bei Hei präsentiert & 28.03.2013 \\ [1ex] \hline & &  \\ [-1.5ex]
4 & Erster Feldtest Stäheli & 31.03.2013 \\ [1ex] \hline & &  \\ [-1.5ex]
5 & Zwischenpräsentation mit Gegenleser und zweiter Prototyp vorgestellt & 26.04.2013 \\ [3.5ex] \hline & &  \\ [-1.5ex]
6 & Komplettsystem Test & 02.05.2013 \\ [1ex] \hline & &  \\ [-1.5ex]
7 & Feature Freeze & 24.05.2013 \\ [1ex] \hline & &  \\ [-1.5ex]
8 & Code Freeze & 31.05.2013 \\ [1ex] \hline & &  \\ [-1.5ex]
9 & Abstract / Broschürentext einreichen &  07.06.2013 \\ [1ex] \hline & &  \\ [-1.5ex]
10 & Abgabe und Präsentation Poster & 14.06.2013 \\ [1ex] 
\caption{Übersicht aller Meilensteine}
\end{longtable} 

Die Gesamtplanung mit den An- und Abwesenheiten der Projektmitgliedern und den Meilensteinen wurde in einer Excel Datei erfasst, diese Datei befindet sich auf der CD. Die aufgewendete Arbeitszeit zu den jeweiligen Arbeitspaketen wurde im Projektverwaltungstool Redmine aufgezeichnet und kann unter \url{http://ita.cnlab.ch/redmine/projects/ba-tourlive} eingesehen werden.
\\
%TODO wie machen wir das mit der Zeiterfassung? Redmine? 

\subsection{Dokumentation}
Für diese Dokumentation wurde das Textsatzprogramm {\LaTeX} verwendet. Es lässt sich optimal mit dem Versionierungssystem \textit{\gls{git}} kombinieren, was für die Zusammenarbeit an der Dokumentation vorteilhaft ist.
\\

Zu jedem Meeting mit dem Professor wurde ein Protokoll geschrieben. Sämtliche Protokolle und zusätzliche Dokumente befinden sich im Word Format auf der CD.